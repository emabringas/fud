\chapter{Trabajos a futuro}
\label{chapter:future:work}

La conclusión de este proyecto abre las puertas a FuDePAN al mundo de la computación voluntaria por lo que es importante avanzar sobre ciertos puntos. Es por ello que a continuación se presentará una lista de trabajos a futuro y funcionalidades extras para este proyecto, que serán dejadas como futuras tareas de la fundación:

\begin{itemize}
\item Instruir y profundizar los conocimientos sobre la administración y configuración de un proyecto BOINC con el fin de lanzar oficialmente un proyecto de computación voluntaria, \textbf{FuDePAN@HOME}, en donde se puedan ejecutar aquellas aplicaciones desarrolladas con FuD-BOINC.

\item Implementar las tareas de validación de resultados como un método específico de FuD-BOINC de tal manera que se pueda realizar validación de resultados redundantes en caso de que la aplicación lo requiera.

\item Extender el rediseño de múltiple jobs presentado en la sección \ref{seccion:multiples:jobunits:clientes} para que el lado cliente de la capa aplicación (L3) sea quien determine la cantidad de trabajos simultáneos que puede recibir.

\item Implementar un screensaver con un logo personalizado de FuDePAN utilizando la API de BOINC para que, al colaborar, cada voluntario pueda observar diferentes detalles de sus trabajos.

\item Agregar la entrega de créditos a los clientes respecto de la cantidad de trabajos procesados.

\end{itemize}
